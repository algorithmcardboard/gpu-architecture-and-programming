\documentclass{article}

\usepackage{tabulary}
\usepackage{hyperref}

\begin{document}
\title{GPU - hw 1}
\author{Anirudhan J. Rajagopalan --- N18824115}
\maketitle

\newpage

\section{Q1}

\begin{tabulary}{\textwidth}{LLLLL}
    \bfseries
    Gpu Model & \bfseries Memory (GB) &\bfseries Num Cores &\bfseries Bandwidth (GB/sec) &\bfseries Year \mdseries \\
    \hline \\
    GTX Titan X~\cite{gpu:titan_x} & 12 GB GDDR5 & 3072 & 336.5 & March, 2015 \\\\
    GTX 980 Ti~\cite{gpu:980_ti} & 6GB GDDR5 & 2816 & 336.5 & June, 2015 \\\\
    GTX 1080~\cite{gpu:gtx_1080} &  8 GB GDDR5X & 2560 & 320 & May, 2016    \\\\
    GTX 1070~\cite{gpu:gtx_1070} & 8 GB GDDR5 & 1920 & 256 & June, 2016 \\\\
    GTX 1060~\cite{gpu:gtx_1060} & 6 GB GDDR5 & 1280 & 192 & July, 2016 \\\\
\end{tabulary}

\section{Q2}
\subsection{Bottlenecks}
The main bottleneck in GPUs is its memory.   While CPU's have evloved to use memory on the order of 72 GB or even 144GBs, GPUs have a maximum of 12GB so far.

\subsection{Bottleneck set to continue}
Based on the table above, we can see that the memory bottleneck can be expected to continue in the near future.


\section{Q3}

\bibliographystyle{plain}
\bibliography{references}

\end{document}
